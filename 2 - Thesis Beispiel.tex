\documentclass[a4paper,12pt]{article}
\usepackage[utf8]{inputenc}
\usepackage{graphicx}
\usepackage{amsmath}
\usepackage{geometry}
\usepackage{booktabs}

\geometry{left=2.5cm, right=2.5cm, top=2.5cm, bottom=2.5cm}

\title{Warum JavaScript in den nächsten 10 Jahren noch mehr an Relevanz gewinnen wird}
\author{Marcus Hartmann}
\date{\today}

\begin{document}

\maketitle

\begin{abstract}
JavaScript ist eine der vielseitigsten Programmiersprachen in der modernen Softwareentwicklung. Diese Arbeit untersucht, warum JavaScript in den kommenden zehn Jahren eine noch wichtigere Rolle spielen wird, indem technologische Trends und Anwendungsbereiche analysiert werden. Eine Prognose des exponentiellen Wachstums wird mithilfe einer mathematischen Formel erstellt.
\end{abstract}

\tableofcontents
\newpage

\section{Einleitung}
JavaScript hat sich in den letzten zwei Jahrzehnten zu einer unverzichtbaren Sprache für die Webentwicklung entwickelt. Mit der Verbreitung von Frameworks wie Angular, React und Vue.js sowie durch die Erweiterung der Nutzung auf den Backend-Bereich (Node.js) hat JavaScript eine zentrale Rolle in der Softwareentwicklung eingenommen.

\section{JavaScript im Kontext der Webentwicklung}
JavaScript war ursprünglich eine clientseitige Sprache, hat jedoch durch die Einführung von ES6 und TypeScript sowie durch Frameworks und Bibliotheken an Leistungsfähigkeit gewonnen. Dank dieser Entwicklungen und der wachsenden Community ist JavaScript heute die meistgenutzte Sprache für Webanwendungen und immer häufiger auch in anderen Anwendungsbereichen vertreten.

\section{Zukunftsaussichten: Warum JavaScript weiterhin dominieren wird}
\subsection{Technologische Trends und die Rolle von JavaScript}
Die Entwicklung hin zu Single-Page-Applications (SPAs), Progressive Web Apps (PWAs) und serverseitigem JavaScript sind nur einige der Trends, die JavaScript in den Vordergrund rücken. Diese Sprache wird außerdem vermehrt in aufstrebenden Bereichen wie IoT, Web3 und Machine Learning eingesetzt.

\subsection{JavaScript im Vergleich zu anderen Sprachen}

\begin{table}[h]
    \centering
    \begin{tabular}{p{2.5cm} p{3.5cm} p{2.5cm} p{4cm} p{2.5cm}}
        \toprule
        \textbf{Sprache} & \textbf{Anwendungsgebiete} & \textbf{Verbreitung} & \textbf{Verfügbarkeit von Tools} & \textbf{Entwickler-freundlichkeit} \\
        \midrule
        JavaScript & Webentwicklung, \newline Backend, IoT, ML & Sehr hoch & Sehr hoch (React, \newline Angular, Node.js) & Hoch \\
        Python     & Data Science, \newline ML, Web, IoT      & Hoch      & Hoch (Django, \newline Flask, TensorFlow) & Sehr hoch \\
        Java       & Backend, \newline Android, Enterprise    & Mittel    & Hoch (Spring, \newline Hibernate) & Mittel \\
        Go         & Cloud, \newline Backend, DevOps          & Mittel    & Mittel (Kubernetes, \newline Docker) & Hoch \\
        \bottomrule
    \end{tabular}
    \caption{Vergleich von JavaScript mit anderen Sprachen}
    \label{tab:vergleich}
\end{table}


Die Tabelle~\ref{tab:vergleich} zeigt, dass JavaScript aufgrund seiner Vielseitigkeit und hohen Verfügbarkeit an Frameworks eine besonders wichtige Sprache bleibt und noch an Bedeutung gewinnen wird.

\section{Datenanalyse: Google Trends für JavaScript}
Um das wachsende Interesse an JavaScript zu verdeutlichen, wurde die Suchanfragen-Statistik von Google Trends ausgewertet. Abbildung~\ref{fig:trends} zeigt das Interesse an JavaScript über die letzten Jahre.

\begin{figure}[h]
    \centering
    \includegraphics[width=0.7\textwidth]{javascript_trends.png} 
    \caption{Interesse an JavaScript laut Google Trends}
    \label{fig:trends}
\end{figure}

Die Grafik zeigt, dass das Interesse an JavaScript kontinuierlich wächst, was durch die Einführung neuer Technologien und Anwendungsbereiche unterstützt wird.

\section{Mathematische Prognose für die Verbreitung von JavaScript}
\subsection{Exponentielles Wachstum der JavaScript-Nutzung}
Das exponentielle Wachstum der JavaScript-Nutzung lässt sich durch die folgende mathematische Gleichung modellieren:

\begin{equation}
    N(t) = N_0 \cdot e^{rt}
\end{equation}

Dabei gilt:
\begin{itemize}
    \item \( N(t) \): Anzahl der JavaScript-Nutzer nach \( t \) Jahren
    \item \( N_0 \): aktuelle Anzahl der JavaScript-Nutzer
    \item \( r \): jährliche Wachstumsrate
    \item \( t \): Zeit in Jahren
\end{itemize}

Diese Formel ermöglicht eine Prognose, wie sich die Anzahl der JavaScript-Nutzer basierend auf einer geschätzten jährlichen Wachstumsrate entwickeln wird.

\section{Fazit und Ausblick}
Zusammenfassend lässt sich sagen, dass JavaScript aufgrund der ständigen Weiterentwicklung, der hohen Entwicklerfreundlichkeit und der breiten Einsatzmöglichkeiten in der Webentwicklung und darüber hinaus in den nächsten zehn Jahren an Relevanz gewinnen wird.

\newpage
\section{Quellenverzeichnis}
\begin{itemize}
    \item W3Techs. (2024). Usage statistics and market share of JavaScript for websites. Abgerufen von \texttt{https://w3techs.com/technologies/details/cp-javascript}
    \item Google Trends. (2024). JavaScript Interest Over Time. Abgerufen von \texttt{https://trends.google.com/}
    \item Stack Overflow. (2024). Developer Survey Results. Abgerufen von \texttt{https://insights.stackoverflow.com/}
    \item GitHub Octoverse Report. (2024). The state of the Octoverse. Abgerufen von \texttt{https://octoverse.github.com/}
\end{itemize}

\end{document}
